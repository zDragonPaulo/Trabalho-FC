\documentclass[a4paper]{article}
% Pacotes necessários
\usepackage[portuguese]{babel}
\usepackage[backend=biber, style=apa, citestyle=apa, language=portuguese]{biblatex}
\usepackage{csquotes}
\addbibresource{Recursos/referencias.bib}

\usepackage{amsmath}
\usepackage{graphicx}
\usepackage{subcaption}
\usepackage{setspace}
\usepackage{siunitx} % Required for alignment
\sisetup{
  round-mode          = places, % Rounds numbers
  round-precision     = 2, % to 2 places
}
\usepackage{enumerate}
\usepackage{enumitem}
\usepackage{amsmath}
\usepackage{karnaugh-map}
\usepackage[section]{placeins}
\usepackage{geometry}
\usepackage{amssymb}
\usepackage{titling}
\usepackage[T1]{fontenc}
\usepackage{float}
\usepackage[hidelinks]{hyperref}
\usepackage{xcolor}
\usepackage{indentfirst}
\usepackage{array}
\usepackage{soul}
\usepackage{afterpage}
\newcolumntype{P}[1]{>{\centering\arraybackslash}p{#1}}
\onehalfspacing


% Comando para criar uma página vazia
\newcommand\myemptypage{
    \null
    \thispagestyle{empty}
    \addtocounter{page}{-1}
    \newpage
}

% Página de título principal
\newcommand{\firsttitlepage}{
    \begin{titlepage}
        \centering
        \vspace*{1cm}
        
        % Logos superior
        \begin{figure}[h!]
            \centering
            \includegraphics[width=6cm]{Recursos/LOGO_IPB} % Substitua pelo caminho da imagem
            \vspace{0.5cm}
        \end{figure}

        % Informações da instituição
        \large\textbf{INSTITUTO POLITÉCNICO DE BEJA} \\
        \large\textbf{Escola Superior de Tecnologia e Gestão} \\
        \large\textbf{Mestrado em Engenharia de Segurança Informática} \\
        \large\textbf{Fundamentos de Cibersegurança} \\
        
        \vspace{2cm}
        
        % Título do projeto
        {\Huge \textbf{Caso 1}} \\
        
        \vspace{1.5cm}
        
        % Autores
        \large Paulo António Tavares Abade - 23919 \\
        
        \vfill
        
        % Logo inferior
        \begin{figure}[h!]
            \centering
            \includegraphics[width=6cm]{Recursos/IPBejaESTIG.jpg} % Substitua pelo caminho da imagem
        \end{figure}
        
        \vspace{1cm}
        
        % Local e data
        {\large Beja, outubro de 2025}
    \end{titlepage}
}

\newcommand{\secondtitlepage}{
    \begin{titlepage}
        \centering
        \vspace*{1cm}
        
        % Informações da instituição
        \large\textbf{INSTITUTO POLITÉCNICO DE BEJA} \\
        \large\textbf{Escola Superior de Tecnologia e Gestão} \\
        \large\textbf{Mestrado em Engenharia de Segurança Informática} \\
        \large\textbf{Fundamentos de Cibersegurança} \\
        
        \vspace{2cm}
        
        % Título do projeto
        {\Huge \textbf{Caso 1}} \\
        
        \vspace{1.5cm}
        
        % Autores
        \large Paulo António Tavares Abade - 23919 \\

        \vspace{2cm}

        % Orientador
        \large Orientadores: Rui Silva \& Rogério Bravo \\
        
        \vfill
        
        % Local e data
        {\large Beja, outubro de 2025}
    \end{titlepage}
}

\begin{document}


\pagenumbering{gobble} % Oculta numeração da página

% Primeira página de título
\firsttitlepage

\secondtitlepage


% Abstract
\section*{\LARGE\textbf{\textit{Resumo}}}

Resolução do trabalho 


\vspace{1cm}
% Keywords
\textbf{Keywords:} cibersegurança
%--------------------------------------------------------------------------------------------------------------------------------------

\section*{\LARGE\textbf{\textit{Abstract}}}

Work Resolution

\vspace{1cm}
% Keywords
\textbf{Keywords:} cybersecurity
\renewcommand{\contentsname}{Índice}       % Título do sumário
\renewcommand{\listfigurename}{Índice de Figuras} % Título da lista de figuras

% Início do conteúdo do relatório
\newpage
\doublespacing
\tableofcontents
\listoffigures
\doublespacing

\newpage
\pagenumbering{arabic}

\section{Grupo I - Professor Rui Silva}
Nesta secção serão respondidas as questões do Grupo I, focando-se na áreas do MITRE ATT\&CK 
lecionadas pelo professor Rui Silva.
\subsection{1.1}
Em resposta à questão 1.1, foram escolhidas para apresentar as áreas de \textit{Malware Protection} e \textit{Incident Coordination}, que podem ser
consideradas como mutualismo/simbiose, uma vez que ambas as áreas trabalham em conjunto para fortalecer a defesa contra ameaças. Caso uma ameaça 
seja detectada, pela área de \textit{Malware Protection}, a área de \textit{Incident Coordination} entra em ação para coordenar a resposta ao incidente,
assegurando que as medidas adequadas sejam tomadas para mitigar o impacto da ameaça.
\subsubsection{Malware Protection}
A proteção de Malware é uma área que tem como objetivo impedir o acesso, a propagação e o impacto de softwares maliciosos (malware) em sistemas informáticos. 
Esta área é efetiva contra ameaças já conhecidas, porém quando se trata de ameaças que utilizam exploits zero-day, ou seja, vulnerabilidades desconhecidas, a eficácia pode ser limitada. 
Mesmo assim, a proteção de malware é uma componente crucial na defesa em profundidade, sendo que esta pode incluir várias técnicas e ferramentas, tais como 
integrar e correlacionar informação de multiplicas fontes estáticos e dinâmicas para detetar com cada vez mais qualidade a presença de malware através da análise 
da sua assinatura.
\subsubsection{Incident Coordination}
A coordenação de incidentes é uma área que se concentra na gestão e resposta a incidentes de segurança informática. Esta área é responsável por coordenar as ações
necessárias para lidar com incidentes, desde a deteção até à resolução. A coordenação de incidentes envolve a comunicação eficaz entre diferentes equipas, para que
as decisões sejam tomadas o mais rapidamente possível para minimizar os danos.

%---------------------------------------------------------------------------------------------------------------------------

\newpage
\renewcommand{\refname}{Bibliografia} % Para artigos
\renewcommand{\bibname}{Bibliografia} % Para livros e relatórios
\addcontentsline{toc}{section}{Bibliografia} % Adiciona a Bibliografia ao índice
\printbibliography

\end{document}
