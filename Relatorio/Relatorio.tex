\documentclass[a4paper]{article}
% Pacotes necessários
\usepackage[portuguese]{babel}
\usepackage[backend=biber, style=apa, citestyle=apa, language=portuguese]{biblatex}
\usepackage{csquotes}
\addbibresource{Recursos/referencias.bib}

\usepackage{amsmath}
\usepackage{graphicx}
\usepackage{subcaption}
\usepackage{setspace}
\usepackage{siunitx} % Required for alignment
\sisetup{
  round-mode          = places, % Rounds numbers
  round-precision     = 2, % to 2 places
}
\usepackage{enumerate}
\usepackage{enumitem}
\usepackage{amsmath}
\usepackage{karnaugh-map}
\usepackage[section]{placeins}
\usepackage{geometry}
\usepackage{amssymb}
\usepackage{titling}
\usepackage[T1]{fontenc}
\usepackage{float}
\usepackage[hidelinks]{hyperref}
\usepackage{xcolor}
\usepackage{indentfirst}
\usepackage{array}
\usepackage{soul}
\usepackage{afterpage}
\newcolumntype{P}[1]{>{\centering\arraybackslash}p{#1}}
\onehalfspacing


% Comando para criar uma página vazia
\newcommand\myemptypage{
    \null
    \thispagestyle{empty}
    \addtocounter{page}{-1}
    \newpage
}

% Página de título principal
\newcommand{\firsttitlepage}{
    \begin{titlepage}
        \centering
        \vspace*{1cm}
        
        % Logos superior
        \begin{figure}[h!]
            \centering
            \includegraphics[width=6cm]{Recursos/LOGO_IPB} % Substitua pelo caminho da imagem
            \vspace{0.5cm}
        \end{figure}

        % Informações da instituição
        \large\textbf{INSTITUTO POLITÉCNICO DE BEJA} \\
        \large\textbf{Escola Superior de Tecnologia e Gestão} \\
        \large\textbf{Mestrado em Engenharia de Segurança Informática} \\
        \large\textbf{Fundamentos de Cibersegurança} \\
        
        \vspace{2cm}
        
        % Título do projeto
        {\Huge \textbf{Trabalho Individual}} \\
        
        \vspace{1.5cm}
        
        % Autores
        \large Paulo António Tavares Abade - 23919 \\
        
        \vfill
        
        % Logo inferior
        \begin{figure}[h!]
            \centering
            \includegraphics[width=6cm]{Recursos/IPBejaESTIG.jpg} % Substitua pelo caminho da imagem
        \end{figure}
        
        \vspace{1cm}
        
        % Local e data
        {\large Beja, outubro de 2025}
    \end{titlepage}
}

\newcommand{\secondtitlepage}{
    \begin{titlepage}
        \centering
        \vspace*{1cm}
        
        % Informações da instituição
        \large\textbf{INSTITUTO POLITÉCNICO DE BEJA} \\
        \large\textbf{Escola Superior de Tecnologia e Gestão} \\
        \large\textbf{Mestrado em Engenharia de Segurança Informática} \\
        \large\textbf{Fundamentos de Cibersegurança} \\
        
        \vspace{2cm}
        
        % Título do projeto
        {\Huge \textbf{Trabalho Individual}} \\
        
        \vspace{1.5cm}
        
        % Autores
        \large Paulo António Tavares Abade - 23919 \\

        \vspace{2cm}

        % Orientador
        \large Orientadores: Rui Silva \& Rogério Bravo \\
        
        \vfill
        
        % Local e data
        {\large Beja, outubro de 2025}
    \end{titlepage}
}

\begin{document}


\pagenumbering{gobble} % Oculta numeração da página

% Primeira página de título
\firsttitlepage

\secondtitlepage


% Abstract
\section*{\LARGE\textbf{\textit{Resumo}}}

Resolução do trabalho 


\vspace{1cm}
% Keywords
\textbf{Keywords:} cibersegurança
%--------------------------------------------------------------------------------------------------------------------------------------

\section*{\LARGE\textbf{\textit{Abstract}}}

Work Resolution

\vspace{1cm}
% Keywords
\textbf{Keywords:} cybersecurity
\renewcommand{\contentsname}{Índice}       % Título do sumário
\renewcommand{\listfigurename}{Índice de Figuras} % Título da lista de figuras

% Início do conteúdo do relatório
\newpage
\doublespacing
\tableofcontents
\listoffigures
\doublespacing

\newpage
\pagenumbering{arabic}

\section{Grupo I - Professor Rui Silva}
Nesta secção serão respondidas as questões do Grupo I, focando-se na áreas do MITRE ATT\&CK 
lecionadas pelo professor Rui Silva.
\subsection{Pergunta}
Em resposta à questão 1.1, foram escolhidas para apresentar as áreas de \textit{Malware Protection} e \textit{Incident Coordination}, que podem ser
consideradas como mutualismo/simbiose, uma vez que ambas as áreas trabalham em conjunto para fortalecer a defesa contra ameaças. Caso uma ameaça 
seja detectada, pela área de \textit{Malware Protection}, a área de \textit{Incident Coordination} entra em ação para coordenar a resposta ao incidente,
assegurando que as medidas adequadas sejam tomadas para mitigar o impacto da ameaça.
\subsubsection{Malware Protection}
A proteção contra malware envolve a implementação de medidas e tecnologias para prevenir, detectar e remover software malicioso que possa comprometer a segurança dos sistemas informáticos. No entanto, esta 
proteção não é infalível, podendo ser contornada por malware que esteja camuflado ou que nunca tenha sido identificado, no caso do último, é conhecido como \textit{Zero-Day Malware}. Esta proteção 
funciona através da análise de padrões comuns em ataques (CAPE), sendo que estes padrões foram identificados através do MAEC (Malware Attribute Enumeration and Characterization), que é um padrão 
para a representação de informações sobre malware, permitindo a troca estruturada de dados entre diferentes ferramentas e sistemas de segurança. O objetivo principal do MAEC é facilitar a 
detecção, análise e resposta a ameaças de malware, promovendo a interoperabilidade entre diferentes soluções de segurança. Este é utilizado pelo Incident Coordination para ajudar a prevenir 
novos ataques com base no que a proteção de malware não conseguiu impedir.
\newpage
\subsubsection{Incident Coordination}
A coordenação de incidentes envolve a gestão e resposta a incidentes de segurança informática que não tenham sido superados pela proteção de malware, garantindo que as ameaças sejam tratadas de forma eficaz e eficiente. Isto inclui a identificação, análise, 
contenção, erradicação e recuperação de incidentes de segurança. A coordenação eficaz de incidentes é crucial para minimizar o impacto das ameaças e garantir a continuidade das operações.
Integrando e correlacionando informação de multiplicas fontes estáticos e dinâmicas, mais conhecidamente como IODEF (Incident Object Description Exchange Format), que é um padrão para a troca estruturada 
de informações sobre incidentes de segurança informática, sendo que isto permite detetar com cada vez mais qualidade a presença de malware através da análise 
da sua assinatura. Isto fica automatizado com o RID (Realtime Inter-network Defense), que é um protocolo que permite a troca automática e segura de informações sobre ameaças. Existem ainda outros 
prótocolos como o TAXII (Trusted Automated eXchange of Indicator Information) que é um protocolo para a troca automatizada de indicadores de ameaças, e o STIX (Structured Threat Information eXpression) 
que é uma linguagem padronizada para a representação de informações sobre ameaças cibernéticas, e estes são complementares sendo que o STIX é o formato que passa pelo TAXII para ser transmitido entre sistemas.


\subsection{Pergunta - MITRE - Projeto de ATT\&CK - Night Dragon}
O projecto \textit{Night Dragon} foi uma campanha de ciberespionagem que visou várias empresas de energia, petróleo e os seus derivados, 
sediadas no Cazaquistão, Taiwan, Grécia e Estados Unidos da América, e a campanha foi descoberta em novembro de 2009 pela McAfee. 
O objetivo principal desta campanha era roubar informações confidenciais e proprietárias relacionadas com a indústria de energia. 
Primeiramente, os atacantes compraram serviços para alojar os servidores que iriam controlar as vítimas e usavam os protocolos HTTP 
como meio de comunicação, pois estes ficavam disfarçados entre o tráfego legítimo. A partir de SQL Injection para obter os dados, 
era utilizado o software \textit{Cain \& Abel} para realizar ataques de brute-force para decifrar os hashes das palavras-passe 
dos administradores de sistemas, conseguindo assim aceder remotamente aos sistemas das vítimas, e com o \textit{zwShell} 
implantado os atacantes podiam executar comandos remotamente. Com isso, os atacantes começaram a obter ficheiros e outras 
informações sensíveis de sistemas comprometidos, enviando-os para os servidores que tinham sido comprados anteriormente.


Ainda utilizaram um RAT (Remote Access Trojan), usando os servidores afetados para fazer ataques a alvos internos através 
de e-mails de spear-phishing que continham anexos maliciosos que, quando abertos, instalavam o RAT nos sistemas das vítimas. 
O alvo principal desta parte do ataque eram portáteis que tinham contas de VPN e que permitiam obter ainda mais acesso aos 
sistemas internos. A estratégia baseava-se em usar ferramentas de roubo de palavras-passe e, ao entrar, deixar um RAT.

Após a investigação, a McAfee concluiu que o grupo responsável pelo \textit{Night Dragon} tinha ligações à China, 
trabalhava entre as 9h e as 17h no horário de Pequim, e que o grupo tinha como alvo empresas específicas, sugerindo 
que a campanha era motivada por interesses económicos e estratégicos. 


\subsection{Pergunta - MITRE - Projeto ATT\&CK - Tática de Initial Access}

A tática de \textit{Initial Access} tem o objetivo de conseguir o primeiro ponto de entrada numa rede e/ou alvo. Existem várias 
técnicas para alcançar este objetivo, sendo a mais conhecida o \textit{Phishing/Spear-Phishing}, que envolve o envio de e-mails, 
SMS ou aplicações que aparentam ser legítimos, mas que contêm links ou anexos maliciosos. No caso do \textit{Enterprise}, qualquer 
funcionário pode ser alvo deste tipo e-mails, enquanto que no \textit{Mobile}, o alvo costuma ser o utilizador final e pode ser 
vitíma através de SMS, aplicações ou chamadas telefónicas. Por fim, no \textit{ICS}, o alvo vai ser o funcionário ou grupo de funcionários 
de uma organização que possua acesso a sistemas industriais, e estes podem ser vítimas através de e-mails ou chamadas telefónicas. 

Existem estas varias de \textit{Phishing/Spear-Phishing} porque cada ambiente tem as suas particularidades, e o atacante deve adaptar-se a cada um deles, 
maximizando as chances de sucesso para cada situação, onde cada alvo tem os seus próprios hábitos e rotinas.
\newpage
\section{Grupo II - Professor Rogério Bravo}

Nesta secção serão respondidas as questões do Grupo II, focando-se na parte da teoria da cibersegurança, 
variando desde os pilares da cibersegurança, intervenção digital forense, os padrões ISO 27000, as três dimensões 
da cibersegurança, entre outros tópicos lecionados pelo professor Rogério Bravo.
\subsection{Pergunta - Os 4 Pilares da Cibersegurança}
Os quatro pilares da cibersegurança são as tecnologias, as pessoas, as organizações e a segurança física. 
Estes pilares sustentam a base em que as ISO 27000 são construídas, e cada um deles desempenha um papel 
crucial na proteção dos sistemas informáticos e dos dados contra ameaças cibernéticas.
%---------------------------------------------------------------------------------------------------------------------------
\subsubsection{Tecnologias}
As tecnologias referem-se às ferramentas, softwares e infraestruturas utilizadas para proteger os sistemas 
informáticos. Isto inclui firewalls, sistemas de deteção de intrusões, antivírus, criptografia e outras 
soluções de segurança que ajudam a prevenir, detectar e responder a ameaças cibernéticas. 
%---------------------------------------------------------------------------------------------------------------------------
\subsubsection{Pessoas}
As pessoas é o pilar que mais impacto tem na cibersegurança, uma vez que representam o elo mais fraco da cibersegurança, 
pois são quem utiliza as tecnologias e quem tem acesso a sistemas e dados sensíveis, e se não forem devidamente treinadas e conscientes das ameaças, podem 
inadvertidamente comprometer a segurança dos sistemas. A formação e a sensibilização dos utilizadores 
são essenciais para garantir que eles compreendam os riscos e adotem práticas seguras.
%---------------------------------------------------------------------------------------------------------------------------
\newpage
\subsubsection{Organizações}
As organizações são responsáveis por estabelecer políticas, procedimentos e práticas de cibersegurança. Isto inclui a 
definição de normas de segurança, a implementação de controles de acesso, a realização de auditorias de segurança e a 
gestão de incidentes. As organizações devem criar uma cultura de segurança que envolva todos os colaboradores e garanta 
a conformidade com as regulamentações e melhores práticas.

\subsubsection{Segurança Física}
A segurança física refere-se à proteção dos ativos físicos, como servidores, data centers e dispositivos de rede, contra
acessos não autorizados, roubos e danos. Isto inclui medidas como controlo de acesso físico, vigilância por vídeo, 
sistemas de alarme e proteção contra desastres naturais. Ou seja, é fundamental que esteja num local seguro e com acesso 
restrito apenas a pessoas autorizadas. Se possível também deve estar protegido contra falhas de energia ou falhas 
de hardware, nomeadamente um disco rígido com defeito.
%---------------------------------------------------------------------------------------------------------------------------

\subsubsection{O pilar na intervenção digital forense}
No contexto da intervenção digital forense, o pilar das pessoas é particularmente crucial, uma vez que 
os profissionais forenses são responsáveis por coletar, analisar e preservar evidências digitais de forma 
ética e legal. Eles devem possuir um profundo conhecimento técnico, bem como uma compreensão das leis e 
regulamentos relacionados com a cibersegurança e a privacidade. Além de que devem seguir rigorosos 
protocolos para garantir a integridade das evidências e evitar contaminação ou adulteração, como por exemplo, 
ao iniciar uma análise forense, impedir que o antívirus faça alterações nos ficheiros, ou seja, colocar o sistema em 
modo de \textit{read-only}, entrando no parte do pilar das tecnologias e ao mesmo tempo entra no pilar das 
organizações, onde estas devem ter políticas e procedimentos claros para a condução de investigações forenses, 
garantindo a conformidade com as leis e regulamentos aplicáveis.
\newpage
%---------------------------------------------------------------------------------------------------------------------------
\subsection{As três dimensões da segurança da informação}
\newpage
\renewcommand{\refname}{Bibliografia} % Para artigos
\renewcommand{\bibname}{Bibliografia} % Para livros e relatórios
\addcontentsline{toc}{section}{Bibliografia} % Adiciona a Bibliografia ao índice
\printbibliography

\end{document}
