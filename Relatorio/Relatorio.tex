\documentclass[a4paper]{article}
% Pacotes necessários
\usepackage[portuguese]{babel}
\usepackage[backend=biber, style=apa, citestyle=apa, language=portuguese]{biblatex}
\usepackage{csquotes}
\addbibresource{Recursos/referencias.bib}

\usepackage{amsmath}
\usepackage{graphicx}
\usepackage{subcaption}
\usepackage{setspace}
\usepackage{siunitx} % Required for alignment
\sisetup{
  round-mode          = places, % Rounds numbers
  round-precision     = 2, % to 2 places
}
\usepackage{enumerate}
\usepackage{enumitem}
\usepackage{amsmath}
\usepackage{karnaugh-map}
\usepackage[section]{placeins}
\usepackage{geometry}
\usepackage{amssymb}
\usepackage{titling}
\usepackage[T1]{fontenc}
\usepackage{float}
\usepackage[hidelinks]{hyperref}
\usepackage{xcolor}
\usepackage{indentfirst}
\usepackage{array}
\usepackage{soul}
\usepackage{afterpage}
\newcolumntype{P}[1]{>{\centering\arraybackslash}p{#1}}
\onehalfspacing


% Comando para criar uma página vazia
\newcommand\myemptypage{
    \null
    \thispagestyle{empty}
    \addtocounter{page}{-1}
    \newpage
}

% Página de título principal
\newcommand{\firsttitlepage}{
    \begin{titlepage}
        \centering
        \vspace*{1cm}
        
        % Logos superior
        \begin{figure}[h!]
            \centering
            \includegraphics[width=6cm]{Recursos/LOGO_IPB} % Substitua pelo caminho da imagem
            \vspace{0.5cm}
        \end{figure}

        % Informações da instituição
        \large\textbf{INSTITUTO POLITÉCNICO DE BEJA} \\
        \large\textbf{Escola Superior de Tecnologia e Gestão} \\
        \large\textbf{Mestrado em Engenharia de Segurança Informática} \\
        \large\textbf{Fundamentos de Cibersegurança} \\
        
        \vspace{2cm}
        
        % Título do projeto
        {\Huge \textbf{Trabalho Individual}} \\
        
        \vspace{1.5cm}
        
        % Autores
        \large Paulo António Tavares Abade - 23919 \\
        
        \vfill
        
        % Logo inferior
        \begin{figure}[h!]
            \centering
            \includegraphics[width=6cm]{Recursos/IPBejaESTIG.jpg} % Substitua pelo caminho da imagem
        \end{figure}
        
        \vspace{1cm}
        
        % Local e data
        {\large Beja, dezembro de 2025}
    \end{titlepage}
}

\newcommand{\secondtitlepage}{
    \begin{titlepage}
        \centering
        \vspace*{1cm}
        
        % Informações da instituição
        \large\textbf{INSTITUTO POLITÉCNICO DE BEJA} \\
        \large\textbf{Escola Superior de Tecnologia e Gestão} \\
        \large\textbf{Mestrado em Engenharia de Segurança Informática} \\
        \large\textbf{Fundamentos de Cibersegurança} \\
        
        \vspace{2cm}
        
        % Título do projeto
        {\Huge \textbf{Trabalho Individual}} \\
        
        \vspace{1.5cm}
        
        % Autores
        \large Paulo António Tavares Abade - 23919 \\

        \vspace{2cm}

        % Orientador
        \large Orientadores: Rui Silva \& Rogério Bravo \\
        
        \vfill
        
        % Local e data
        {\large Beja, dezembro de 2025}
    \end{titlepage}
}

\begin{document}


\pagenumbering{gobble} % Oculta numeração da página

% Primeira página de título
\firsttitlepage

\secondtitlepage


% Abstract
\section*{\LARGE\textbf{\textit{Resumo}}}

Resolução do trabalho individual de Fundamentos de Cibersegurança, respondendo às questões propostas pelos professores Rui Silva 
e Rogério Bravo, abrangendo tópicos como o MITRE ATT\&CK, incident coordination, pilares da cibersegurança, dimensões da 
segurança da informação e o conceito de governança.


\vspace{1cm}
% Keywords
\textbf{Keywords:} cibersegurança
%--------------------------------------------------------------------------------------------------------------------------------------

\renewcommand{\contentsname}{Índice}       % Título do sumário


% Início do conteúdo do relatório
\newpage
\doublespacing
\tableofcontents
\doublespacing

\newpage
\pagenumbering{arabic}

\section{Grupo I - Professor Rui Silva}
Nesta secção serão respondidas as questões do Grupo I, focando-se na áreas do MITRE ATT\&CK 
lecionadas pelo professor Rui Silva.

\subsection{Pergunta - Apresentação do Incident Coordination}
Em resposta à questão 1.1, foi escolhido o tema \textit{Incident Coordination} (\cite{incident_coordination}).
A coordenação de incidentes consiste na gestão centralizada e na resposta a incidentes de segurança informática 
que ultrapassam as medidas preventivas. Engloba as fases de identificação, triagem e análise, contenção, erradicação, 
recuperação e lições aprendidas. Uma coordenação eficaz minimiza o impacto das ameaças e assegura a continuidade das operações.

Para suportar estas atividades procede-se à correlação de informação proveniente de múltiplas fontes (logs, 
SIEM, deteções de endpoints, telemetria de rede e inteligência de ameaças). Existem formatos e protocolos que 
facilitam a troca e automatização desta informação: 

\begin{enumerate}
    \item IODEF (Incident Object Description Exchange Format) é um formato para descrever e trocar relatórios de incidentes;
    \item o STIX (Structured Threat Information eXpression) define um modelo para representar indicadores e contexto de ameaças; 
    \item o TAXII (Trusted Automated eXchange of Indicator Information) fornece os mecanismos de transporte/partilha de objetos 
    STIX entre sistemas; 
    \item RID (Realtime Inter-network Defense) refere-se a mecanismos para a troca automática, em tempo quase real, de informações 
    de defesa entre redes. 
\end{enumerate}
Estes padrões são complementares e permitem melhorar a deteção, correlação e resposta a incidentes, 
incluindo automação de ações quando apropriado.

\newpage
\subsection{Pergunta - MITRE - Projeto de ATT\&CK - Night Dragon}
O projecto \textit{Night Dragon} (\cite{dragon}) foi uma campanha de ciberespionagem que teve como alvo várias empresas de energia, petróleo e 
os seus derivados, sediadas no Cazaquistão, Taiwan, Grécia e Estados Unidos da América. A campanha foi descoberta em novembro 
de 2009 pela McAfee. 

O objetivo principal desta campanha foi roubar informações confidenciais e proprietárias relacionadas com a indústria de energia. 
Primeiramente, os atacantes compraram serviços para alojar os servidores que viriam controlar as vítimas e usaram os protocolos HTTP 
como meio de comunicação, de modo a ficarem disfarçados entre o tráfego legítimo. A partir de ataques de SQL Injection foram obtidos os dados, 
e esses, eram submetidos ao software \textit{Cain \& Abel} para realizar ataques de brute-force para decifrar os hashes das palavras-passe 
dos administradores de sistemas. 

Após decifrar os hashs, os atacantes conseguiram aceder remotamente aos sistemas das vítimas, e com o \textit{zwShell} 
implantado os atacantes podiam executar comandos remotamente e começaram a obter ficheiros e outras 
informações sensíveis dos sistemas comprometidos, enviando-os para os servidores que tinham sido comprados anteriormente.


Ainda utilizaram um RAT (Remote Access Trojan), usando os servidores afetados para fazer ataques a alvos internos através 
de e-mails de \textit{spear-phishing} que continham anexos maliciosos que, quando abertos, instalavam o RAT nos sistemas das vítimas. 
O alvo principal desta parte do ataque eram portáteis que tinham contas de VPN e que permitiam obter ainda mais acesso aos 
sistemas internos. A estratégia baseava-se em usar ferramentas de roubo de palavras-passe e, ao entrar, deixar um RAT.

Após a investigação, a McAfee concluiu que o grupo responsável pelo \textit{Night Dragon} tinha ligações à China, 
trabalhava entre as 9h e as 17h no horário de Pequim, e que o grupo tinha como alvo empresas específicas, sugerindo 
que a campanha era motivada por interesses económicos e estratégicos. 

\newpage

\subsection{Pergunta - MITRE - Projeto ATT\&CK - Tática de Initial Access}

Foi escolhida a tática de \textit{Initial Access}(\cite{mitre_initial_access}) do MITRE ATT\&CK, que é a primeira fase do ciclo de vida de um ataque cibernético, 
e será comparada em três ambientes distintos: \textit{Enterprise}, \textit{Mobile} e \textit{ICS} (Industrial Control Systems). 
O ambiente \textit{Enterprise} refere-se a redes corporativas e sistemas informáticos utilizados por empresas, o ambiente \textit{Mobile} 
refere-se a dispositivos móveis como smartphones e tablets, e o ambiente \textit{ICS} refere-se a sistemas de controlo industrial utilizados 
em infraestruturas críticas, como por exemplo: redes elétricas, fábricas e instalações de tratamento de água.

A tática de \textit{Initial Access} tem o objetivo de conseguir o primeiro ponto de entrada numa rede e/ou alvo. Existem várias 
técnicas para alcançar este objetivo, sendo a mais conhecida o \textit{Phishing/Spear-Phishing}, que envolve o envio de e-mails, 
SMS ou aplicações que aparentam ser legítimos, mas que contêm links ou anexos maliciosos. No caso do \textit{Enterprise}, qualquer 
funcionário da empresa pode ser alvo deste tipo e-mails, enquanto que no \textit{Mobile}, o alvo costuma ser o utilizador final e pode ser 
vitíma através de SMS, aplicações ou chamadas telefónicas. Por fim, no \textit{ICS}, o alvo vai ser o funcionário ou grupo de funcionários 
de uma organização que possua acesso a sistemas industriais, e estes podem ser vítimas através de e-mails ou chamadas telefónicas. 

Existem diversas variantes de \textit{Phishing} porque cada ambiente tem as suas particularidades, e o atacante deve adaptar-se a cada um deles, 
maximizando as chances de sucesso para cada situação, onde cada alvo tem os seus próprios hábitos e rotinas. No caso do \textit{Enterprise}, o \textit{Phising} 
tem quatro variantes principais: \textit{Spear-Phishing Attachment}, onde o atacante envia um e-mail com um anexo malicioso que, quando aberto, 
instala malware no sistema da vítima, o \textit{Spear-Phishing Link}, onde o atacante envia um e-mail com um link malicioso que, quando clicado, 
redireciona a vítima para um site falso ou instala malware, o \textit{Spear-Phishing via Service}, onde o atacante utiliza serviços de mensagens 
instantâneas ou redes sociais para enviar mensagens maliciosas, e por fim, o \textit{Spear-Phishing Voice}, onde o atacante faz chamadas telefónicas
fingindo ser uma entidade confiável para enganar a vítima e obter informações sensíveis ou instalar malware. 

\newpage
Existem ainda outras técnicas dentro do \textit{Initial Access}, como a exploração de serviços públicos expostos como API's para o 
caso do \textit{Enterprise}, serviços de cloud no caso do \textit{Mobile} e por fim, no caso do \textit{ICS} existe a possibilidade 
de um gateway exposto que possa ser explorado. 

Outra técnica possível é a instalação de softwares de terceiros que possam ter sido comprometidos, como por exemplo, bibliotecas ou atualizações 
de software no caso do \textit{Enterprise}, aplicações móveis no caso do \textit{Mobile} e softwares de monitorização ou controlo no caso do \textit{ICS}.

Existem ainda outras técnicas que podem ser utilizadas para obter acesso inicial, como a Drive-by Compromise, onde o atacante compromete um site legítimo 
para infectar visitantes com malware, sendo que afeta os três ambientes que estão a ser mencionados, pois basta um utilizador com previlégios de administrador 
aceder ao site comprometido e por erro, clicar num link ou descarregar um ficheiro malicioso mesmo que sem o seu conhecimento, o atacante consegue obter acesso inicial.
Existe ainda a técnica de \textit{Removale Media}, onde o atacante utiliza dispositivos de armazenamento removíveis, como pen's ou discos externos, que ao serem 
inseridos no sistema da vítima, instalam malware ou permitem o acesso remoto. Esta técnica é mais comum no ambiente \textit{Enterprise} e \textit{ICS},
onde os sistemas podem não ter proteções adequadas contra dispositivos externos, enquanto que no ambiente \textit{Mobile} é menos comum, pois os dispositivos móveis
geralmente não permitem a execução automática de malware a partir de dispositivos externos. Para o caso do \textit{Enterprise}, existe ainda a técnica de \textit{Hardware Additions} 
onde o atacante insere um hardware que pode ser um \textit{keyLogger} que regista as teclas digitadas e envia para o atacante, ou um dispostivo que insere 
comandos maliciosos na vítima ao ser ligado ao sistema. Esta técnica é diferente do \textit{Removale Media}, pois a arma é o hardware em si, e não o conteúdo que este possa ter e 
não necessita de ser lido ou executado para que o ataque seja bem-sucedido.

Em suma, a tática de \textit{Initial Access} do MITRE ATT\&CK apresenta várias técnicas que podem ser adaptadas a diferentes ambientes,
permitindo aos atacantes obter acesso inicial a sistemas e redes de forma eficaz.


\newpage
\section{Grupo II - Professor Rogério Bravo}

Nesta secção serão respondidas as questões do Grupo II, focando-se na parte da teoria da cibersegurança, 
variando desde os pilares da cibersegurança, intervenção digital forense, os padrões \textit{ISO 27000}, as três dimensões 
da cibersegurança, entre outros tópicos lecionados pelo professor Rogério Bravo. As respostas serão baseadas em
\subsection{Pergunta - Os 4 Pilares da Cibersegurança}
Os quatro pilares da cibersegurança são as tecnologias, as pessoas, as organizações e a segurança física. 
Estes pilares sustentam a base em que as \textit{ISO 27000} são construídas, e cada um deles desempenha um papel 
crucial na proteção dos sistemas informáticos e dos dados contra ameaças cibernéticas.
%---------------------------------------------------------------------------------------------------------------------------
\subsubsection{Tecnologias}
As tecnologias referem-se às ferramentas, softwares e infraestruturas utilizadas para proteger os sistemas 
informáticos. Isto inclui firewalls, sistemas de deteção de intrusões, antivírus, criptografia e outras 
soluções de segurança que ajudam a prevenir, detectar e responder a ameaças cibernéticas. 
%---------------------------------------------------------------------------------------------------------------------------
\subsubsection{Pessoas}
As pessoas é o pilar com mais impacto na cibersegurança, uma vez que representam o elo mais fraco da cibersegurança, 
pois são quem utilizam as tecnologias e quem tem acesso a sistemas e dados sensíveis. E se não forem devidamente treinadas e conscientes das ameaças, podem 
inadvertidamente comprometer a segurança dos sistemas. A formação e a sensibilização dos utilizadores 
são essenciais para garantir que eles compreendam os riscos e adotem práticas seguras.
%---------------------------------------------------------------------------------------------------------------------------
\newpage
\subsubsection{Organizações}
As organizações são responsáveis por estabelecer políticas, procedimentos e práticas de cibersegurança. Isto inclui a 
definição de normas de segurança, a implementação de controles de acesso, a realização de auditorias de segurança e a 
gestão de incidentes. As organizações devem criar uma cultura de segurança que envolva todos os colaboradores e garanta 
a conformidade com as regulamentações e melhores práticas.

\subsubsection{Segurança Física}
A segurança física refere-se à proteção dos ativos físicos, como servidores, \textit{data centers} e dispositivos de rede, contra
acessos não autorizados, roubos e danos. Isto inclui medidas como controlo de acesso físico, vigilância por vídeo, 
sistemas de alarme e proteção contra desastres naturais. Ou seja, é fundamental que os dispositivos estejam num local seguro e com acesso 
restrito apenas a pessoas autorizadas. Se possível também devem estar protegido contra falhas de energia ou falhas 
de \textit{hardware}, como por exemplo, um disco rígido com defeito.
%---------------------------------------------------------------------------------------------------------------------------
\subsubsection{Relação entre Pilares e Princípios}
Cada pilar da segurança da informação contribui de forma distinta para assegurar os princípios da segurança, sendo estes a Confidencialidade, Integridade, Disponibilidade, 
e o Não Repúdio. A seguir está uma descrição de como cada pilar se relaciona com estes princípios:

\begin{description}
    \item[Pessoas:] 
    Este pilar é determinante para a \textbf{Confidencialidade} e \textbf{Autenticidade}. É através da consciencialização dos utilizadores que se previnem fugas 
    de informação (engenharia social) e se garante que as credenciais de acesso são utilizadas apenas pelos legítimos proprietários.
    
    \item[Segurança Física:] 
    Atua primariamente na garantia da \textbf{Disponibilidade} e \textbf{Integridade física} dos ativos. Assegura que o hardware e as infraestruturas (energia, 
    climatização) permanecem operacionais e protegidos contra roubo ou destruição por desastres naturais.
    \newpage
    \item[Tecnológico:] 
    É o principal motor técnico para a \textbf{Integridade} e o \textbf{Não Repúdio}. Utiliza mecanismos lógicos como \textit{hashing}, assinaturas digitais e 
    sistemas de \textit{logging} para validar que os dados não foram alterados e para registar inequivocamente a autoria das ações. Também reforça a 
    Confidencialidade através da criptografia.
    
    \item[Organizacional:] 
    Este pilar é \textbf{transversal a todos os princípios}. É responsável pela governação, definindo através de políticas e normas o nível de Confidencialidade, Integridade e Disponibilidade exigido para cada tipo de informação, servindo de base legal para a aplicação de sanções em caso de violação (suportando o aspeto jurídico do Não Repúdio).
\end{description}
%---------------------------------------------------------------------------------------------------------------------------
\subsubsection{O pilar na intervenção digital forense}
No contexto da intervenção digital forense, o pilar das pessoas é particularmente crucial, uma vez que 
os profissionais forenses são responsáveis por coletar, analisar e preservar evidências digitais de forma 
ética e legal. Eles devem possuir um profundo conhecimento técnico, bem como uma compreensão das leis e 
regulamentos relacionados com a cibersegurança e a privacidade. Além de que devem seguir rigorosos 
protocolos para garantir a integridade das evidências e evitar contaminação ou adulteração, como por exemplo, 
ao iniciar uma análise forense, impedir que o antívirus faça alterações nos ficheiros, ou seja, colocar o sistema em 
modo de \textit{read-only}, entrando na parte do pilar das tecnologias e ao mesmo tempo no pilar das 
organizações, onde estas devem ter políticas e procedimentos claros para a condução de investigações forenses, 
garantindo a conformidade com as leis e regulamentos aplicáveis. Tudo isto demonstra também a importância dos princípios 
da Integridade e Não Repúdio na intervenção digital forense.
\newpage
%---------------------------------------------------------------------------------------------------------------------------
\subsection{As três dimensões da segurança da informação}
Existem três dimensões principais para garantir a segurança da informação, sendo elas a segurança 
física, a segurança lógica e a segurança humana. Todas estas dimensões estão ligadas entre si, e estão de acordo 
com a \textit{SEGNAC4}, sendo esta a \textit{Resolução do Conselho de Ministros 5/1990}, onde são estabelecidas normas e procedimentos 
que visam garantir a segurança da informação em sistemas informáticos, especialmente aqueles utilizados por entidades
públicas.
%---------------------------------------------------------------------------------------------------------------------------
\subsubsection{Segurança Física}
A segurança física envolve a proteção dos dispositivos físicos, como servidores, \textit{data centers} e dispositivos de rede, 
contra acessos não autorizados, roubos e danos. Isto inclui medidas como controlo de acesso físico, vigilância por vídeo, 
sistemas de alarme e proteção contra desastres naturais. No contexto da SEGNAC4, a segurança física é fundamental para garantir 
o acesso restrito a áreas sensíveis e a proteção dos equipamentos contra ameaças que possam roubar ou danificar a informação armazenada.
%---------------------------------------------------------------------------------------------------------------------------
\subsubsection{Segurança Lógica}
A segurança lógica baseia-se na proteção dos sistemas informáticos e das redes contra acessos não autorizados digitalmente, onde as medidas 
incluem autenticação, criptografia, \textit{firewalls}, entre outros métodos que possam limitar o acesso a estes sistemas. 
No âmbito da \textit{SEGNAC4}, é essencial implementar estas medidas para garantir a confidencialidade e integridade da informação classificada que está 
armazenada ou a ser transmitida através de sistemas informáticos.
%---------------------------------------------------------------------------------------------------------------------------
\subsubsection{Segurança Humana}
A segurança humana refere-se à formação e sensibilização dos utilizadores para garantir que compreendam os riscos 
ao quais estão exposos e adotem práticas seguras, como a criação de palavras-passe fortes, o reconhecimento de tentativas de \textit{phishing}
e não partilhar informações sobre o trabalho, especialmente em ambientes de acesso público. No contexto da \textit{SEGNAC4}, a segurança humana é 
crucial para prevenir vulnerabilidades que possam ser exploradas por atacantes, já que o fator humano é frequentemente a maior fraqueza 
dos sistemas informáticos.
\newpage 
%---------------------------------------------------------------------------------------------------------------------------
\subsection{O Conceito de Governança}
A Governança, ao ser apresentada neste curso, refere-se a saber \textbf{quando e como alguém} acedeu a um sistema informático e a partir de que 
utilizador e com que permissões foi feito esse acesso, sendo uma monitorização que permite injeção de informação nos \textit{SIEM} 
(Security Information and Event Management) para permitir assim identificar mais fácilmente os ataques e os seus padrões, permitindo ainda uma reconstrução 
de como foi realizado o ataque. Mantém ainda uma gestão de dispositivos e registo de logs. Estes logs devem conter o endereço IP, a data e 
hora do acesso (em hora, minuto, segundos e fuso horário) e o porto de comunicação utilizado. A Governança deve ainda incluir assinaturas de 
ataques, segmentação de redes para evitar a propagação de incidentes e segregação de funções para impedir o comprometimento de dados.

Este conceito coincide com a \textit{Resolução do Conselho de Ministros 41/2018}, que promove o reforço das capacidades de monitorização, 
deteção, resposta a incidentes e incentiva a implementação de medidas técnicas para aumentar a resiliência das infraestruturas 
críticas e a promoção de uma cultura de segurança transversal em todos os setores, sejam estes públicos, privados ou académicos. 

É possível afirmar, com toda a certeza, que a Governança apresentada no curso está perfeitamente alinhada com os objetivos estratégicos da 
\textit{Resolução do Conselho de Ministros 41/2018}, no que diz respeito ao reforço das capacidades de monitorização, deteção e resposta a incidentes, tal 
como definido nas normas ISO 27000, onde todos estes conceitos e normas são complementares entre si e visam o mesmo objetivo final, que é a proteção dos 
sistemas informáticos e dos dados contra ameaças cibernéticas em Portugal.
%---------------------------------------------------------------------------------------------------------------------------
\newpage
\renewcommand{\refname}{Bibliografia} % Para artigos
\renewcommand{\bibname}{Bibliografia} % Para livros e relatórios
\addcontentsline{toc}{section}{Bibliografia} % Adiciona a Bibliografia ao índice
\printbibliography

\end{document}
